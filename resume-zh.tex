%%
%% Copyright (c) 2018-2019 Weitian LI <wt@liwt.net>
%% CC BY 4.0 License
%%
%% Created: 2018-04-11
%%

% Chinese version
\documentclass[zh]{resume}
\usepackage{multicol}
% Adjust icon size (default: same size as the text)
\iconsize{\Large}

% File information shown at the footer of the last page
\fileinfo{%
  \faCopyright{} 2017 -- 2021,Zhonghao Sun \hspace{0.5em}
  \creativecommons{by}{4.0} \hspace{0.5em}
  \githublink{syvshc}{resume} \hspace{0.5em}
  \faEdit{} \today
}

\name{忠豪}{孙}

% \keywords{BSD,Linux,Programming,Python,C,Shell,DevOps,SysAdmin}
\keywords{Mathimatics,PDE,LDG,LaTeX,Cube}

% \tagline{\icon{\faBinoculars}} <position-to-look-for>}
% \tagline{<current-position>}

\photo{2cm}{avatar.jpg}

\profile{
  \mobile{186-0460-9792}
  \email{syvshc@foxmail.com}
  \github{syvshc} \\
  \university{哈尔滨工业大学}
  \degree{理学 \textbullet 学士}
  \birthday{1998-12-09}
  \iconlink{\faFileAlt}{https://syvshc.github.io/}{个人博客: \texttt{syvshc.github.io}}
  \address{黑龙江 \textbullet 哈尔滨}
  % Custom information:
  % \icontext{<icon>}{<text>}
}

\begin{document}
\makeheader

%======================================================================
% Summary & Objectives
%======================================================================
{\onehalfspacing\hspace{2em}%
% 物理学专业(射电天文方向)直博研究生,有扎实的物理、数学与统计学基础,
% 擅长数据建模与分析,热衷计算机和网络技术,
% 有 10 年的 Linux 和 BSD 使用经验,熟练掌握 Shell、Python 和 C 语言编程。
% 积极实践自由开源精神,
哈尔滨工业大学数学学院信息与计算科学系本科毕业生,有扎实的数学专业课的基础,曾获人民奖学金;
可以使用 Python 与 MATLAB 进行数值分析;
能熟练使用 \LaTeX 进行论文排版;对知识充满敬畏,对计算机以及数值充满兴趣,对生活充满热爱。
在 \link{https://github.com/syvshc}{GitHub} 上分享多个项目。
\par}

%======================================================================
\sectionTitle{教育背景}{\faGraduationCap}
%======================================================================
% \begin{educations}
%   \education%
%     {2017.09}%
%     [2021.06]%
%     {哈尔滨工业大学}%
%     {数学学院}%
%     {信息与计算科学}%
%     {学士}
% \end{educations}
\begin{tblr}{
  colspec = {rl},
  vline{2},
}
  2017/09 -- 2021/06 & 哈尔滨工业大学 \textbullet 数学学院 \\
  & 信息与计算科学 \textbullet 学士
\end{tblr}

%======================================================================
\sectionTitle{基本能力}{\faWrench}
%======================================================================
% \begin{competences}
%   \comptence{编程}{%
    
%   }
%   \comptence{工具}{%
%     Git,\LaTeX
%   }
%   \comptence{语言}{
%     \textbf{英语} --- 已取得 CET6 证书
%   }
% \end{competences}

\begin{tblr}{
  colspec = {rl},
  column{1} = {cmd = \textbf}
}
  编程  & Python,MATLAB \\
  工具  & git,\LaTeX\\
  语言  & \textbf{英语} --- 已取得 CET6 证书
\end{tblr}


%======================================================================
\sectionTitle{课程成绩}{\faBookOpen}
%======================================================================

\begin{tblr}{
  colspec       = {llll},
  column{1}     = {6em, cmd = \textbf},
  column{3}     = {10em, cmd = \textbf},
  column{2, 4}  = {6em},
  cell{1}{1}    = {c = 4}{l}
}
  平均学分绩: 82 分, 专业排名: 8/24 \\
  数学分析 II   & 85 分   & 高等代数 I        & 97 分   \\
  高等代数 II   & 89 分   & 复变函数          & 82 分   \\
  数值逼近      & 92.1 分 & 泛函分析基础      & 85.3 分 \\
  概率论        & 99 分   & 微分方程数值方法  & 78 分 \\
\end{tblr}

%======================================================================
\sectionTitle{科研经历}{\faCode}
%======================================================================
% \begin{experiences}
% \experience[2017/10]{2018/05}{哈尔滨工业大学大一年度项目}[作为课题《由魔方引发的数学思考》的负责人,获得理学院一等奖]
% \separator{.5ex}
% \experience[2020/11]{2021/06}{哈尔滨工业大学本科毕业设计 (论文)}[《间断有限元方法求解线性 KdV 方程的最优误差估计》获得 91.7 分的成绩,位于数学学院前 15\%]  
% \end{experiences}

\begin{tblr}{
  colspec = {rl},
  vline{2},
  hline{3} = {1ex, white},
  cell{1, 3}{2} = {cmd = \textbf}
}
  2017/10 -- 2018/05  & 哈尔滨工业大学大一年度项目 \\
                      & 作为课题《由魔方引发的数学思考》的负责人,获得理学院一等奖。\\
  2020/11 -- 2021/06  & 哈尔滨工业大学本科毕业设计 (论文)\\
                      & 《间断有限元方法求解线性 KdV 方程的最优误差估计》获得 91.7 分的成绩,\\
                      & 位于数学学院前 15\%.
\end{tblr}

%======================================================================
\sectionTitle{综合素质}{\faFish}
%======================================================================

\begin{itemize}
  \item 于 2018 年担任哈尔滨工业大学爱魔魔术社舞台部副部长;
  \item 于 2019 年担任哈尔滨工业大学茶楼相声社秘书处部长以及社长,并多次在学校的大型晚会上表演相声节目;
  \item 于 2019 年举办了一场 \LaTeX 的介绍讲座;
  \item 于 2019 年秋季学期与同学一起使用 \LaTeX 对\textbf{泛函分析基础}的课程笔记进行整理,并分享至 \link{https://github.com/syvshc/2019Fall_FA}{Github},广受好评;
  \item 于 2021 年对 \TeX\,Live 的包管理器的文档进行了翻译,发布在 \link{https://github.com/syvshc/tlmgr-intro-zh-cn}{Github} 与 \link{https://ctan.org/pkg/tlmgr-intro-zh-cn}{CTAN};
  \item 于 2021 年在清华大学的 \link{https://github.com/tuna/THU-Beamer-Theme}{THU -- Beamer -- Theme} 模板的基础上进行功能的增减,制作了适用于哈尔滨工业大学的 \link{https://github.com/syvshc/HITBeamer}{HITBeamer 模板},并使用它进行了毕业论文的答辩;
  \item 于 2021 年搭建了我的个人博客: \link{https://syvshc.github.io/}{\texttt{syvshc.github.io}},并在博客上分享所学的知识;
  \item 于 2017 -- 2021 年之间多次进行高中数学的家教,被辅导对象成绩均获得明显的提升。
\end{itemize}
\end{document}
